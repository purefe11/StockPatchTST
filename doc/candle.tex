% https://www.overleaf.com/
\documentclass[tikz, border=10pt]{standalone}
\usepackage{kotex}
\usepackage{fontspec}
\setmainfont{NanumGothic}
\usepackage{tikz}
\usetikzlibrary{positioning}

\begin{document}

\begin{tikzpicture}[
    every node/.style={font=\small},
    candle/.style={draw=black, line width=0.7pt},
    red/.style={fill=red!60},
    blue/.style={fill=blue!60},
    label/.style={font=\footnotesize}
]

% 고가/저가 공통 위치
\def\high  {2.2}
\def\low   {-2}

% 양봉 (시가 < 종가)
\def\openBull  {-0.8}
\def\closeBull {1}

% 음봉 (시가 > 종가)
\def\openBear  {1}
\def\closeBear {-0.8}

% 양봉
\draw[candle, red] (-4,\closeBull) rectangle (-3,\openBull);
\draw[candle] (-3.5,\high) -- (-3.5,\closeBull);
\draw[candle] (-3.5,\openBull) -- (-3.5,\low);

% 음봉
\draw[candle, blue] (3,\openBear) rectangle (4,\closeBear);
\draw[candle] (3.5,\high) -- (3.5,\openBear);
\draw[candle] (3.5,\closeBear) -- (3.5,\low);

% 수평선만 먼저 그림
\foreach \y in {\high, \openBull, \closeBull, \openBear, \closeBear, \low} {
    \draw[dashed, gray!60] (-6,\y) -- (6,\y);
}

% 라벨 (왼쪽: 양봉 기준 / 오른쪽: 음봉 기준)
\node[label] at (-6.4,\high)       {고가};
\node[label] at (-6.4,\openBull)   {시가 (양봉)};
\node[label] at (-6.4,\closeBull)  {종가 (양봉)};
\node[label] at (-6.4,\low)        {저가};

\node[label] at (6.4,\high)        {고가};
\node[label] at (6.4,\openBear)    {시가 (음봉)};
\node[label] at (6.4,\closeBear)   {종가 (음봉)};
\node[label] at (6.4,\low)         {저가};

% 구성 설명
\node at (0,1.6)    {윗꼬리};
\node at (0,0.1)    {몸통};
\node at (0,-1.4)   {아랫꼬리};

% 하단 텍스트
\node at (-3.5,-2.5) {양봉};
\node at (3.5,-2.5)  {음봉};

% 제목
%\node[font=\large] at (0,2.8) {캔들 차트};

\end{tikzpicture}

\end{document}
